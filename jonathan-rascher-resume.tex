% Copyright © 2010-11 Jonathan Rascher.

% Use a basic, minimal resume class, and boost the font size a tiny bit.
\documentclass[11pt]{simplecv}

% TeX uses relatively large margins by default. This leads to very nice pages
% when the document being typeset consists on large blocks of flowing text. On a
% resume, however, it looks rather too sparse. We fix that by shrinking the
% margins a bit.
\usepackage[letterpaper,hmargin=0.84in,vcentering]{geometry}

% Set some nice fonts. You can download or purchase them here:
% * <http://friedrichalthausen.de/?page_id=411>
% * <http://www.exljbris.com/fertigo.html>
%
% XXX: XeTeX doesn't automatically find Vollkorn styles for some reason.
\usepackage{adforn,fontspec}
\setmainfont[BoldFont={*-Bold},
             ItalicFont={*-Italic},
             BoldItalicFont={*-BoldItalic},
             Ligatures=TeX]
            {Vollkorn}
\newfontfamily\fertigopro[Ligatures=TeX]{Fertigo Pro}

% Tell the simplecv class about our cool fonts.
\headerfont{\fertigopro}
\titlefont{\fertigopro\LARGE}
\sectionfont{\fertigopro\large}
\subsectionfont{\fertigopro}
\topiclabelfont{\fertigopro\itshape}
\topictitlefont{\fertigopro\itshape}

% Customize the page footers to add the total page count (to inform the reader
% if they lose a page) and add some neat ornaments.
\usepackage{fancyhdr,lastpage}
\pagestyle{fancy}
\fancyhf{}
\cfoot{%
  \fertigopro\footnotesize%
  {\normalsize\adfflatleafoutlineleft}%
  \hspace{0.5em}%
  \thepage\ of \pageref*{LastPage}%
  \hspace{0.5em}%
  {\normalsize\adfflatleafoutlineright}}
\renewcommand*\headrulewidth{0pt}
\renewcommand*\footrulewidth{0pt}

% Load miscellaneous packages (e-TeX helpers, LaTeX fixes, and support for
% multiple columns of text).
\usepackage{etoolbox,fixltx2e,multicol}

% Set up PDF extras.
\usepackage[dvipsnames,usenames]{color}
\usepackage{hyperref}
\hypersetup{
  colorlinks,
  linkcolor=RawSienna,
  citecolor=RawSienna,
  urlcolor=RawSienna,
  pdfpagemode=UseNone,
  pdftitle=Jonathan Rascher's Resume,
  pdfauthor=Jonathan Rascher,
  pdfstartview=FitH,
  pdfprintscaling=None}

% Give our typography a little love.
\newcommand*\amp{{\fertigopro\itshape\&}}
\newcommand*\fracslash{\raisebox{-0.3ex}{/}\kern-0.1em}
\newcommand*\nbds[1][-]{\nobreakdashes#1\hspace{0pt}}
\newcommand*\CPP{%
  C\nolinebreak\kern-0.05em%
  \raisebox{0.12ex}{+\nolinebreak\kern-0.08em+}}
\newcommand*\CS{C\nolinebreak\kern-0.05em\#}
\newcommand*\FS{F\nolinebreak\kern-0.05em\#}

% Widen the left topic column to accommodate long item labels.
\renewcommand*\topicmargin{0.345\columnwidth}

% Tweak spacing between list items to make it tighter in general, but looser
% than usual between "groups" of items, i.e., add some additional space before
% items with label arguments.
\appto\itemize{\setlength\itemsep{1pt}}

\makeatletter
  \pretocmd\@topic@item{%
    \setlength\itemsep{6.5pt}%
    \if@noitemarg%
      \setlength\itemsep{0pt}%
    \fi}
\makeatother

% XXX: Vollkorn has overly tight kerning between the letters `m` and `n` and
%      the apostrophe/right single quote character. This makes phrases like
%      "Dean's List" look awkward and unbalanced. We hack our way around this
%      issue by manually kerning these pairs using XeTeX character classes.
\makeatletter
  \let\preamble@vollkornfamily\f@family

  \XeTeXinterchartokenstate=1

  \newXeTeXintercharclass\preamble@apostrophecc
  \newXeTeXintercharclass\preamble@vollkornkernhackcc

  \XeTeXcharclass`\'=\preamble@apostrophecc
  \XeTeXcharclass`\m=\preamble@vollkornkernhackcc
  \XeTeXcharclass`\n=\preamble@vollkornkernhackcc

  \XeTeXinterchartoks\preamble@vollkornkernhackcc\preamble@apostrophecc={%
    \ifdefstrequal{\f@family}{\preamble@vollkornfamily}{%
      \kern-0.05em}{}}
\makeatother

\begin{document}
  % Print the resume masthead.
  \leftheader{%
    <\href{mailto:jon@bcat.name}{jon@bcat.name}> \\
    +1 630 917 2228}

  \rightheader{%
    PO Box 135 \\
    Wayne, IL 60184-0135}

  \title{\addfontfeature{LetterSpace=2.6}Jonathan Rascher}
  \maketitle

  % The primary content of my resume follows.
  \begin{topic}
    \item[Bachelor of Science] Computer Science \amp\ Mathematics---North Central College.

    \item[Academic Minor] Interactive Media Studies---North Central College.

    \item[] Member of Alpha Lambda Delta and Pi Mu Epsilon honor societies.

    \item Grade-point average 4.0\fracslash 4.0 after 105 credit hours.

    \item Anticipated graduation date: June 2012.
  \end{topic}

  \section{Academic Awards \amp\ Distinctions}
  \begin{topic}
    \item[Academic Yr.\ 2009--10] Entrant in William Lowell Putnam Mathematical Competition.

    Member of ACCA programming contest advanced team.

    \item[Academic Yr.\ 2010] Recipient of Mary Anice Seybold Prize in Mathematics.

    Participant in Google FUSE summer undergraduate retreat.

    \item[Academic Yr.\ 2009] Recipient of Presidential Scholarship.

    Recipient of Science Scholarship.

    Recipient of Phi Theta Kappa Scholarship.

    \item[Fall 2008--Winter 2011] Dean's List (North Central College).

    \item[Fall 2007--Fall 2008] Dean's List (Elgin Community College).

    \item Took first college classes as a high school sophomore participating in an early-admissions program at Elgin Community College.
  \end{topic}

  \newpage

  \section{Programming \amp\ Markup Languages}
  \begin{multicols}{3}
    \begin{itemize}
      \item x86 Assembly
      \item Adobe Flash
      \item Bourne shell
      \item C/\CPP
      \item \CS
      \item CSS
      \item \FS
      \item HTML/XHTML
      \item \LaTeX
      \item Java
      \item JavaScript
      \item PHP
      \item Python
      \item Visual Basic .NET
      \item Visual Basic Classic
    \end{itemize}
  \end{multicols}

  \section{Relevant Concepts \amp\ Coursework}
  \begin{multicols}{2}
    \begin{itemize}
      \item Computer graphics in OpenGL
      \item Computer system concepts
      \item Data structures \amp\ algorithms
      \item Digital \amp\ film photography
      \item Discrete mathematics \amp\ proofs
      \item Linux administration
      \item Robotics on Parallax Javelin Stamp
      \item Single \amp\ multivariate calculus
      \item Undergraduate real analysis
      \item Web development
    \end{itemize}
  \end{multicols}

  \section{Work \amp\ Volunteer Experience}
  \begin{topic}
    \item[Jan. 2011--March 2011] Preceptor: Computer Science II---North Central College.
    \item[June 2010--Sept. 2010] Co\nbds op developer in DIS division---Argonne National Laboratory.
    \item[March 2010--June 2010] Preceptor: Computer Animation with Flash---North Central College.
    \item[Sept. 2009--March 2010] Web application developer---North Central College.
    \item[Sept. 2004--March 2009] Education volunteer---Cosley Zoo.
    \item[Summer 2007, 2008] Food runner (2007), Cashier (2008)---Kane Country Cougars.
    \item[Sept. 2003--Aug. 2007] Teen volunteer---Bartlett Public Library.
  \end{topic}
\end{document}
